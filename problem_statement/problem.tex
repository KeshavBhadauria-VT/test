\problemname{Getting to bed}

\illustration{.3}{bed.jpg}{Source: \href{https://images.unsplash.com/photo-1582582621959-48d27397dc69?ixlib=rb-1.2.1&ixid=MnwxMjA3fDB8MHxwaG90by1wYWdlfHx8fGVufDB8fHx8&auto=format&fit=crop&w=1769&q=80}{Photo by obtained on Unsplash}}


Paul has just moved in, but there are boxes everywhere, however,
it's midnight and he wants to get to bed. The moving company
gives you the dimensions of the room. Some boxes are too heavy to
move without help, and they fit perfectly in the room, so its impossible to move diagonal. Given this
information, is it possible to get to your bed without having to
move the ``heavy boxes?''

The room is given via a coordinate system, as seen below:
\illustration{.3}{FirstExample.jpg}{
    Photo by Keshav Bhadauria
}

The red boxes, are the boxes that are too heavy to move, while the green represents where the bed is. The door will always be at the starting position (0,0).

Here's an example where the moving company left a path to the bed:

\illustration{.3}{SecondExample.jpg}{
    Photo by Keshav Bhadauria
}
The green arrows represents one of the paths Paul can take.


\section*{Input}

The input begins with a single line containing there integers, $n$ ($1 < n < 100$) which tells the length, $m$ ($1 < m < 100$) that tell the width of the room, and $k$ which tells how many unmovable boxes there are. 
After that, there is another line containing $k$ number of integers pairs representing the coordinate pairs containing where the unmovable boxes 
boxes are located. Lastly, there is one last line containing another integer pair (%f% %g%) showing where the bed is located. 

Paul will always start at the position (0,0), and there will be no heavy boxes located at that spot.

\section*{Output}

Output \texttt{SLEEPING} if Paul can reach his bed, otherwise output \texttt{IMPOSSIBLE}.

